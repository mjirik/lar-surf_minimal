
\section{Appendix}

Here we provide, for the sake of readers, a list of symbols used in the paper, and a small set of related definitions. 
A description of public dataset used in the experiments, and 
a coding example of computation of the sparse $[\partial_3]$ boundary matrix in Julia is finally given.

\subsection{Symbol list}

\begin{description}
\item[$\mathcal{I}$]  three-dimensional medical image
\vspace{-2.5mm}\item[$\ell_1, \ell_2, \ell_3$]  dimensions of image
\vspace{-2.5mm}\item[$\mathcal{S}$]  segment: a subset of voxels from image segmentation
\vspace{-2.5mm}\item[$\mathcal{B}$] 3D image brick
\vspace{-2.5mm}\item[$size$] lateral dimension of cubic brick $\mathcal{B}$
\vspace{-2.5mm}\item[$n$]	number of bricks $\mathcal{B}$ (jobs) in $\mathcal{I}$
\vspace{-2.5mm}\item[[$\partial_\mathcal{B}$]]  boundary matrix for brick $\mathcal{B}$
\vspace{-2.5mm}\item[$C_p$] linear (vector) space of $p$-chains
\vspace{-2.5mm}\item[$\nu\in C_p$] $p$-chain
\vspace{-2.5mm}\item[[$\nu$]] coordinate representation (binary vector) of  $\nu$
\vspace{-2.5mm}\item[$\E^3$] Euclidean 3-space
\end{description}


\subsection{Definitions}
\begin{description}
\item[Boundary model] Closed manifold surface of the boundary of a solid model
\vspace{-2.5mm}\item[CSC]  Compressed Sparse Column format for sparse matrices
\vspace{-2.5mm}\item[Global coordinates]  Integer linear coordinates of $\mathcal{I}$
\vspace{-2.5mm}\item[Local coordinates] Integer linear coordinates of $\mathcal{B}$
\vspace{-2.5mm}\item[Cartesian coordinates] Integer triples $(i,j,k)$ one-to-one with voxels
\vspace{-2.5mm}\item[Voxels] Individual elements in 3D image (3-cells)
\vspace{-2.5mm}\item[$p$-chain] Formal linear combination of $p$-cells with coefficients in $\{0,1\}$
\vspace{-2.5mm}\item[Coord. repr.]  Binary vector (for $p$-chains) or binary matrix (for chain operators)
\vspace{-2.5mm}\item[Quad]	Geometric quadrilateral; convex polygon with four vertices
\vspace{-2.5mm}\item[Foreground voxel] Individual element of a segment $\mathcal{S}$
\vspace{-2.5mm}\item[Segment]	Subset of voxels resulting from image segmentation
\end{description}

\subsection{3D-Ircadb Dataset}

For perfomance analysis the public dataset from Research Institute against Digestive Cancer (IRCAD) \cite{ircadb} was used. 
The table 
% \ref{tab:ircad1} and 
\ref{tab:ircad2} describe the dataset.

% \begin{table}
% \begin{tabular}{rrrrrr}
\toprule
 ID &  z-resolution [mm] &  xy-resolution [mm] &  obj. voxels &  size xy &  size z \\
\midrule
  1 &               1.60 &            0.570000 &      2865131 &      512 &     129 \\
  2 &               1.60 &            0.782000 &      1648024 &      512 &     172 \\
  3 &               1.25 &            0.625000 &      2375079 &      512 &     200 \\
  4 &               2.00 &            0.742188 &      1132427 &      512 &      91 \\
  5 &               1.60 &            0.782000 &      2124505 &      512 &     139 \\
  6 &               1.60 &            0.782000 &      1828493 &      512 &     135 \\
  7 &               1.60 &            0.782000 &      1461944 &      512 &     151 \\
  8 &               1.60 &            0.561000 &      3215090 &      512 &     124 \\
  9 &               2.00 &            0.873047 &      1265420 &      512 &     111 \\
 10 &               1.60 &            0.736000 &      1871804 &      512 &     122 \\
 11 &               1.60 &            0.720000 &      1692716 &      512 &     132 \\
 12 &               1.00 &            0.679688 &      3341433 &      512 &     260 \\
 13 &               1.60 &            0.671000 &      2063109 &      512 &     122 \\
 14 &               1.60 &            0.720000 &      1633641 &      512 &     113 \\
 15 &               1.60 &            0.782000 &      1389572 &      512 &     125 \\
 16 &               1.60 &            0.698000 &      2717185 &      512 &     155 \\
 17 &               1.60 &            0.743000 &      2106497 &      512 &     119 \\
 18 &               2.50 &            0.742188 &      1220564 &      512 &      74 \\
 19 &               4.00 &            0.703125 &       583208 &      512 &     124 \\
 20 &               2.00 &            0.808594 &      1359697 &      512 &     225 \\
\bottomrule
\end{tabular}

% \caption{Ircad dataset description, \cite{ircadb}}
% \label{tab:ircad1}
% \end{table}

\begin{table}[h!]
\centering
\begin{tabular}{lrrrrr}
\toprule
{} &  z-resolution [mm] &  xy-resolution [mm] &   obj. voxels &  size xy &      size z \\
\midrule
% count &           20.00000 &           20.000000 &  2.000000e+01 &     20.0 &   20.000000 \\
min   &            1.00000 &            0.561000 &  5.832080e+05 &    512.0 &   74.000000 \\
mean  &            1.77750 &            0.725141 &  1.894777e+06 &    512.0 &  141.150000 \\
50\%   &            1.60000 &            0.739094 &  1.760604e+06 &    512.0 &  127.000000 \\
% std   &            0.60273 &            0.077233 &  7.206126e+05 &      0.0 &   44.088756 \\
% 25\%   &            1.60000 &            0.693422 &  1.382103e+06 &    512.0 &  121.250000 \\
% 75\%   &            1.70000 &            0.782000 &  2.187148e+06 &    512.0 &  152.000000 \\
max   &            4.00000 &            0.873047 &  3.341433e+06 &    512.0 &  260.000000 \\
\bottomrule
\end{tabular}

\caption{Ircad dataset description \cite{ircadb}. It contains 20 Computed Tomography images of abdomen with 
manually segmented tissues.}
\label{tab:ircad2}
\end{table}


\subsection{Basic operations in LAR}
\label{sec:lar-example}

\vspace{10pt}
\noindent\underline{
Boundary matrices for grids of cubes:
}\vspace{0.2em}\newline 
% NOTE this is new 
We give here the full Julia code for the algebraic computation of $\partial_3$ matrix, for a very little grid of unit
3-cubes. Due to the simplicity of the cells (voxels = cubes), a sufficient (geom,top) pair is given below
as (\texttt{V,CV}), where \texttt{CV} is an array of arrays of \texttt{Float64} indices of grid cubes.

% julia> V
% 3x24 Array{Float64,2}:
% julia> CV
% 6-element Array{Array{Int64,1},1}:
\begin{Verbatim}[fontsize=\footnotesize]
julia> using LinearAlgebraicRepresentation, SparseArrays
julia> Lar = LinearAlgebraicRepresentation
julia> V, CV = Lar.cuboidGrid([3,2,1])
julia> V
0.0 0.0 0.0 0.0 0.0 0.0 1.0 1.0 1.0 1.0 1.0 1.0 2.0 2.0 2.0 2.0 2.0 2.0 3.0 3.0 3.0 3.0 3.0 3.0
0.0 0.0 1.0 1.0 2.0 2.0 0.0 0.0 1.0 1.0 2.0 2.0 0.0 0.0 1.0 1.0 2.0 2.0 0.0 0.0 1.0 1.0 2.0 2.0
0.0 1.0 0.0 1.0 0.0 1.0 0.0 1.0 0.0 1.0 0.0 1.0 0.0 1.0 0.0 1.0 0.0 1.0 0.0 1.0 0.0 1.0 0.0 1.0
julia> CV
[[ 1, 2, 3, 4, 7, 8, 9,10], [ 3, 4, 5, 6, 9,10,11,12], [ 7, 8, 9,10,13,14,15,16],
 [ 9,10,11,12,15,16,17,18], [13,14,15,16,19,20,21,22], [15,16,17,18,21,22,23,24]
\end{Verbatim}

\vspace{10pt}
\noindent\underline{
Face and Edge Data generation:
}\vspace{0.2em}\newline 
%TODO describe CV2FV and CV2EV
In the following, we provide the functions for generating the face data \texttt{FV} (vertex indices in faces) with function  \texttt{CV2FV} and edge data \texttt{EV} (vertex indices in edges) with function \texttt{CV2EV} from cell data \texttt{CV}. 
% In particular,  and  functions apply to all the 3-cells in \texttt{CV} the pattern of reference to vertices used by faces and edges of the single 3-cube:



% \footnotesize
%\small]
\begin{Verbatim}[fontsize=\footnotesize]
function CV2FV( v:: Array{ Int64 } )
    return faces = [[v[1], v[2], v[3], v[4]], [v[5], v[6], v[7], v[8]],
                    [v[1], v[2], v[5], v[6]], [v[3], v[4], v[7], v[8]],
                    [v[1], v[3], v[5], v[7]], [v[2], v[4], v[6], v[8]]]
end
function CV2EV( v:: Array{ Int64 } )
    return edges = [[v[1],v[2]], [v[3],v[4]], [v[5],v[6]], [v[7],v[8]], [v[1],v[3]], [v[2],v[4]],
                    [v[5],v[7]], [v[6],v[8]], [v[1],v[5]], [v[2],v[6]], [v[3],v[7]], [v[4],v[8]]]
end
\end{Verbatim}

\vspace{10pt}
\noindent\underline{
Characteristic matrices:
}\vspace{0.2em}\newline 
The function \texttt{K} transforms an array of arrays ($\mathtt{VV, EV, FV, CV}$) into a sparse binary characteristic matrix
($\mathtt{M_0, M_1, M_2, M_3}$). A Julia sparse matrix needs three arrays I, J, Vals of rows, columns, values of non-zeros:

% VV = [[v] for v=1:size(V, 2)]
% [1],[2],[3],[4],[5],[6],[7],[8],[9],[10],[11],[12],[13],[14],[15], [16],[17],
% [18],[19],[20],[21],[22],[23],[24]
% [1],[2],[3],[4],[5],[6],[7],[8],[9],[10],[11],[12],[13],[14],[15], [16],[17],
% [18],[19],[20],[21],[22],[23],[24]
\begin{Verbatim}[fontsize=\footnotesize]
VV = [[v] for v=1:size(V, 2)];
FV = collect(Set{Array{Int64,1}}(vcat(map(CV2FV, CV)...)))
[[13,15,19,21], [1,2,3,4], [7,9,13,15], [13,14,15,16], [7,8,13,14], [1,2,7,8], [2,4,8,10], [7,8,9,10], 
 [3,5,9,11], [8,10,14,16], [15,16,21,22], [9,11,15,17], [3,4,5,6], [17,18,23,24], [11,12,17,18], 
 [1,3,7,9], [3,4,9,10], [9,10,15,16], [4,6,10,12], [13,14,19,20], [9,10,11,12], [15,16,17,18], 
 [19,20,21,22], [15,17,21,23], [16,18,22,24], [21,22,23,24], [10,12,16,18], [5,6,11,12], [14,16,20,22]]

EV = collect(Set{Array{Int64,1}}(vcat(map(CV2EV, CV)...)))
[[15,17], [16,22], [6,12], [17,23], [18,24], [4,10], [3,4], [13,15], [11,12], [9,15], [13,19],
 [1,7], [5,11], [5,6], [12,18], [8,14], [15,21], [17,18], [1,3], [2,4], [16,18], [2,8], [21,23],
 [20,22], [1,2], [14,16], [10,16], [13,14], [19,21], [7,13], [9,10], [23,24], [11,17], [21,22],
 [3,9], [3,5], [9,11], [7,9], [14,20], [7,8], [22,24], [19,20], [8,10], [15,16], [10,12], [4,6]]

function K(CV)
    I = vcat( [ [k for h in CV[k]] for k =1: length(CV) ]...);
    J = vcat(CV ...);
    Vals = Int8[1 for k=1: length(I)];
    return SparseArrays.sparse(I,J,Vals)
end
VV = [[k] for k=1:size(V,2)];
M0 = K(VV); M1 = K(EV); M2 = K(FV); M3 = K(CV);
\end{Verbatim}

\vspace{10pt}
\noindent\underline{
Boundary matrices:
}\vspace{0.2em}\newline 
The boundary matrices between non-oriented chain spaces are computed by sparse matrix multiplication
followed by matrix filtering, produced in Julia by the broadcast of vectorized integer division ($.\div$):

% TODO this is strange the M_1 is M_1 after multiplication?
%$\partial_1 =  \mathtt{M_0} * \mathtt{M'_1} = \mathtt{M'_1}$
%
%$\partial_2 =  \left(\mathtt{M_1} * \mathtt{M'_2}\right) .\div \mathtt{sum(M_1, dims=2)}$
%
%$\partial_3 =  \left(\mathtt{M_2} * \mathtt{M'_3}\right) .\div \mathtt{sum(M_2, dims=2)}$


\begin{Verbatim}[fontsize=\footnotesize]
# This code is working with Julia 1.2
partial_1 = M0 * M1'  
partial_2 = div.((M1 * M2'), 2)
s = sum(M2, dims=2)
partial_3 = (M2 * M3') ./ s
partial_3 = div.(partial_3, 1)
\end{Verbatim}

