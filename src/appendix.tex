
\section{Appendix}

Here we provide, for the sake of readers, a list of symbols used in the paper, and a small set of related definitions. 
A description of public dataset used in the experiments, and 
a coding example of computation of the sparse $[\partial_3]$ boundary matrix in Julia is finally given.

\subsection{Symbol list}

\begin{description}
\item[$\mathcal{I}$]  three-dimensional medical image
\vspace{-2.5mm}\item[$\ell_1, \ell_2, \ell_3$]  dimensions of image
\vspace{-2.5mm}\item[$\mathcal{S}$]  segment: a subset of voxels from image segmentation
\vspace{-2.5mm}\item[$\mathcal{B}$] 3D image brick
\vspace{-2.5mm}\item[$size$] lateral dimension of cubic brick $\mathcal{B}$
\vspace{-2.5mm}\item[$n$]	number of bricks $\mathcal{B}$ (jobs) in $\mathcal{I}$
\vspace{-2.5mm}\item[[$\partial_\mathcal{B}$]]  boundary matrix for brick $\mathcal{B}$
\vspace{-2.5mm}\item[$C_p$] linear (vector) space of $p$-chains
\vspace{-2.5mm}\item[$\nu\in C_p$] $p$-chain
\vspace{-2.5mm}\item[[$\nu$]] coordinate representation (binary vector) of  $\nu$
\vspace{-2.5mm}\item[$\E^3$] Euclidean 3-space
\end{description}


\subsection{Definitions}
\begin{description}
\item[Boundary model] Closed manifold surface of the boundary of a solid model
\vspace{-2.5mm}\item[CSC]  Compressed Sparse Column format for sparse matrices
\vspace{-2.5mm}\item[Global coordinates]  Integer linear coordinates of $\mathcal{I}$
\vspace{-2.5mm}\item[Local coordinates] Integer linear coordinates of $\mathcal{B}$
\vspace{-2.5mm}\item[Cartesian coordinates] Integer triples $(i,j,k)$ one-to-one with voxels
\vspace{-2.5mm}\item[Voxels] Individual elements in 3D image (3-cells)
\vspace{-2.5mm}\item[$p$-chain] Formal linear combination of $p$-cells with coefficients in $\{0,1\}$
\vspace{-2.5mm}\item[Coord. repr.]  Binary vector (for $p$-chains) or binary matrix (for chain operators)
\vspace{-2.5mm}\item[Quad]	Geometric quadrilateral; convex polygon with four vertices
\vspace{-2.5mm}\item[Foreground voxel] Individual element of a segment $\mathcal{S}$
\vspace{-2.5mm}\item[Segment]	Subset of voxels resulting from image segmentation
\end{description}

\subsection{3D-Ircadb Dataset}

For perfomance analysis the public dataset from Research Institute against Digestive Cancer (IRCAD) \cite{ircadb} was used. 
The table 
% \ref{tab:ircad1} and 
\ref{tab:ircad2} describe the dataset.

% \begin{table}
% \input{input/dataset_ircad}
% \caption{Ircad dataset description, \cite{ircadb}}
% \label{tab:ircad1}
% \end{table}

\begin{table}[h!]
\centering
\input{input/dataset_ircad_desc}
\caption{Ircad dataset description \cite{ircadb}. It contains 20 Computed Tomography images of abdomen with 
manually segmented tissues.}
\label{tab:ircad2}
\end{table}


\subsection{Basic operations in LAR}
\label{sec:lar-example}

\input{examples_appendix}